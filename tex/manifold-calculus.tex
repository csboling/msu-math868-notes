\section{Calculus on manifolds}
Let $M$ be a smooth manifold and $p \in M$, with
$C_p^\infty(M)$ the set of germs of smooth functions at $p$, i.e. the
$\mathbb{R}$-algebra of equivalence classes of pairs $(U, f)$, where
U is open, $f: U \to \mathbb{R}$ is smooth and $p \in U$, where two
such pairs are equivalent when
$\exists p \in U \subset U_1 \cap U_2$ with $f_1 = f_2$ on $U$.

Point derivations of $C_p^\infty(M)$ are $\mathbb{R}$-linear maps
$D: C_p^\infty(M) \to \mathbb{R}$ such that
$D(fg)(p) = D(f) g(p) + f(p) D(g)$.

The tangent space $T_pM$ of $M$ at $p$ is the $\mathbb{R}$-vector space of
point derivations of $C_p^\infty(M)$.

\begin{defn}[Differential]
The \emph{differential} of a smooth map $F: N \to M$ at $p$ is a
smooth map
$$
(F_\ast)_p = F_{\ast,p} = D_p F : T_p N \to T_{F(p)} M.
$$
For any $X_p \in T_pN$, $g \in C_{F(p)}^\infty M$, this acts as
$$
F_{\ast,p}(X_p)(g) \triangleq X_P(g \circ F).
$$
\end{defn}

\begin{prop}
  \begin{enumerate}
    \item{
      Given $F: N \to M$ and $G: M \to P$,
      $$
        (G \circ F)_{\ast, p}
      = G_{\ast, F(p)} \circ F_{\ast, p}.
      $$
    }
    \item{
      The differential of $\id_M$ is $\id_{T_pM}$.
    }
  \end{enumerate}
\end{prop}
\begin{proof}
  \begin{enumerate}
    \item{
      Let $X_p \in T_pM$ and $g \in C_{f(p)}^\infty(P)$. Then
      \begin{align*}
         G_{\ast, f(p)} \circ F_{\ast, p} (X_p) g
      &= G_{\ast, f(p)} (F_{\ast, p} (X_p)) g \\
      &= F_{\ast p}(X_p) (g \circ G) \\
      &= X_p(g \circ G \circ F) \\
      &= (G \circ F)_{\ast, p}(X_p)(g).
      \end{align*}
    }
    \item{
      $$
        \id_{\ast,p}(X_p)(g)
      = X_p(g \circ \id_M)
      = X_p(g).
      $$
    }
  \end{enumerate}
\end{proof}

\begin{remark}
For $p \in U \subset M$ with $U$ open,
$C_p^\infty(U) = C_p^\infty(M)$, so $T_pU = T_pM$.
\end{remark}

\begin{corol}
If $F: N \to M$ is a local diffeomorphism, then
$F_{\ast, p} : T_p N \to T_{f(p)} M$ is an isomorphism of vector spaces.
\end{corol}
\begin{proof}
If $F$ has a $C^\infty$ inverse $F^{-1} : M \to N$ at $p$, then
$$
    (F \circ F^{-1})_{\ast, F(p)}
  = (\id_M)_{\ast, F(p)}
  = \id_{T_{F(p)}}
$$
and likewise $(F^{-1} \circ F)_{\ast, p} = \id_{T_pN}$.
\end{proof}

\subsection{Basis for $T_pM$}
Let $(U, x^1, \dots, x^m)$ be a coordinate chart about $p$
and $\phi$ be a chart to $(\mathbb{R}^m, r^1, \dots, r^m)$.
Recall that $T_{\phi(p)} \mathbb{R}^m$ has basis
$\left.\frac{\partial}{\partial r^i}\right|_{\phi(p)}$.
Then form
$\left\{ \left.\frac{\partial}{\partial x^i}\right|_p \right\}_{i=1}^m$.
Since $\phi$ is a diffeomorphism onto its image,
$\phi_\ast : T_pM \to T_{\phi(p)} \mathbb{R}^m$ is an isomorphism. We
compute
\begin{align*}
   \phi_{\ast,p}
     \left(
       \left.
         \frac{\partial}{\partial x^i}
       \right|_p
     \right)(g)
&= \left.
     \frac{\partial}{\partial x^i}
   \right|_p
     (g \circ \phi) \\
&= \frac{\partial (g \circ \phi \circ \phi^{-1})}
        {\partial r^i}
   (\phi(p)) \\
&= \frac{\partial g}{\partial r^i}(\phi(p)) \\
&= \left.\frac{\partial}{\partial r^i}\right|_{\phi(p)} g.
\end{align*}
Therefore
$$
  \phi_\ast
    \left(
      \left.
        \frac{\partial}{\partial x^i}
      \right|_p
    \right)
= \left.
    \frac{\partial}{\partial r^i}
  \right|_{\phi(p)}.
$$
This shows that
$\left\{ \left.\frac{\partial}{\partial x^i}\right|_p
\right\}_{i=1}^m$
is an induced basis for $T_pM$.

\begin{corol}
$\dim T_p M = \dim M = m$. Consequently
$\mathbb{R}^n$ and $\mathbb{R}^m$ are not diffeomorphic if
$n \neq m$. They are not homeomorphic either, but this is more
difficult to show.
\end{corol}

\subsection{Change of coordinates}
Let $p \in M$ and $(U, x^1, \dots, x^m)$, $(V, y^1, \dots, y^m)$ be
coordinates about $p$, with bases $\{\partial_{x^i}|_p\}$ and
$\{\partial_{y_i}|_p\}$. Then we can write
$$
  \left.
    \frac{\partial}{\partial x^j}
  \right|_p
= \sum_{k=1}^m
    a_j^k
    \left.\frac{\partial}{\partial y^k}\right|_p
$$
for some unique coefficients $a_j^k$.

Evaluating both sides on $y^i: U \cap V \to \mathbb{R}$ gives
\begin{align*}
   \sum_{k=1}^n
     a_j^k
     \frac{\partial y^i}{\partial y^k}(p)
&= \sum_k a_j^k \delta_k^i = a_j^i
\end{align*}
so that
$$
  \left.\frac{\partial}{\partial x^j}\right|_p
= \sum_{i=1}^m
    \frac{\partial y^i}{\partial x^j}
    \left.
      \frac{\partial}{\partial y^i}
    \right|_p.
$$

\subsection{Differential in coordinates}
Let $F: N \to M$ with a chart $(U, x, \phi)$ at
$p$ and $(V, y, \psi)$ at $F(p)$. We see that
$$
  F_{\ast,p}
  \left(
    \left.
      \frac{\partial}{\partial x^j}
    \right|_p
  \right)
= \sum_{k=1}^n
    a_m^k
    \left.
      \frac{\partial}{\partial y^k}
    \right|_{F(p)}
$$
for some $a_m^k$. To determine $a_m^k$, let both sides act as a
derivation on $y^i: V \to \mathbb{R}$ so that
\begin{align*}
   F_{\ast, p}
     \left(
       \left.
         \frac{\partial}{\partial x^j}
       \right|_p
     \right)(y^i)
&= \left.
     \frac{\partial}{\partial x^j}
   \right|_p
     (y^i \circ F)
 = \frac{\partial F^i}{\partial x^j}(p)
\end{align*}
whereas
$$
  \left(
    \sum_{k=1}^m
      a_j^k
      \left.
        \frac{\partial}{\partial y^k}
      \right|_{F(p)}
  \right)
    (y^i)
= \sum_{k=1}^n
    a_j^k
    \frac{\partial y^i}{\partial y^k}
= a^i_j.
$$
Therefore in coordinates $F_{\ast,p} = J(F)(p).$

\begin{theorem}[Coordinate-free inverse function theorem]
$F: N \to M$ is a local diffeomorphism at $p$ if and only if
$F_{\ast,p}$ is invertible.
\end{theorem}

\section{Curves}
\begin{defn}[Curve]
A \emph{curve} on a manifold $M$ is a smooth map
$c: (a, b) \to M$ from an interval to the manifold.
\end{defn}

The basis of the tangent space $T_{t_0} (a, b)$ is given by
$$
  \left.\frac{\partial}{\partial t}\right|_{t_0}
= \left.\frac{\dif}{\dif t}\right|_{t_0},
$$
so the \emph{velocity vector} to $c$ at time $t = t_0$ is
$$
           c^\prime(t_0)
\triangleq c_{\ast, t_0}
             \left(
               \left.\frac{\dif}{\dif t}\right|_{t_0}
             \right)
\in T_{c(t_0)} M.
$$

\subsection{Notation}
Let $c: (a, b) \to \mathbb{R}$ be a curve. We have two types of
derivative:
\begin{itemize}
  \item{
    the ordinary derivative from calculus ($c$ is a real function) given by
    $$
      \dot{c}(t_0)
    = \lim_{h \to 0}
        \frac{c(t_0 + h) - c(t_0)}{h}
    \in \mathbb{R}
    $$
  }
  \item{
    the velocity vector (i.e. total derivative) treating $c$ as a
    curve
    $$
      c^\prime(t_0)
    = c_{\ast, t_0}
        \left(
          \left.\frac{\dif}{\dif t}\right|_{t_0}
        \right)
    \in T_{c(t_0)} M.
    $$
  }
\end{itemize}
These notions are related since we can write
$$
  c^\prime(t_0)
= \alpha(t_0)
  \left.\frac{\dif}{\dif r}\right|_{c(t_0)}
$$
for some $\alpha(t_0)$ since $T_{c(t_0)}\mathbb{R}$ is a
one-dimensional vector space.
To determine this, evaluate both sides on the coordinate function
$r: \mathbb{R} \to \mathbb{R}$ to see
\begin{align*}
   c^\prime(t_0) r
&= c_{\ast, t_0}
     \left(
       \left.\frac{\dif}{\dif t}\right|_{t_0}
     \right) r
 = \left.\frac{\dif}{\dif t}\right|_{t_0} (r \circ c) \\
&= \left.\frac{\dif}{\dif t}\right|_{t_0} (c(t))
 = \dot{c}(t_0)
\end{align*}
whereas
$$
  \alpha(t_0)
  \left.\frac{\dif}{\dif r}\right|_{c(t_0)} r
= \alpha(t_0)
$$
so that $\alpha(t_0)$ is precisely the classical derivative.

\subsection{Velocity vectors for curves}
Writing $c^i(t) = r^i(c(t))$, we have
$$
  c(t)
= (c^1(t), \dots, c^n(t)) \in \mathbb{R}^n
$$
so that
$$
  c^\prime(t)
= \sum_i
    a^i(t)
    \left.\frac{\partial}{\partial r^i}\right|_t.
$$
This gives $\dot{c}^j(t) = a^j(t)$ so that
$$
  c^\prime(t)
= \sum_i
    (\dot{c}^i)(t)
    \left.\frac{\partial}{\partial r^i}\right|_t.
$$
In calculus, we write
$$
  c^\prime(t)
= \langle
    (\dot{c^1})(t), \dots, (\dot{c^n})(t)
  \rangle
$$

Every tangent vector is a velocity of many curves.
\begin{lemma}
Let $M$ be a manifold, $p \in M$, and $X_p \in T_pM$.
Then there is a curve
$$
c: (-\varepsilon, \varepsilon) \to M
$$
with $c(0) = p$, $c^\prime(0) = X_p$.
\end{lemma}

\begin{proof}
Let $(U, \phi) = (U, x^1, \dots, x^n)$ be a coordinate chart centered
at $p$, i.e. $\phi(p) = 0 \iff x^1(p) = \cdots = x^n(p) = 0$. We write
$$
  X_p
= \sum_i
    a^i \left.\frac{\partial}{\partial x^i}\right|_p
\in T_p M
$$
so that
\begin{align*}
   \phi_{\ast, p}(X_p)
&= \sum_i
     a^i \phi_{\ast, p}
     \left(
       \left.\frac{\partial}{\partial x^i}\right|_p
     \right) \\
&= \sum_i
     a^i
     \left.\frac{\partial}{\partial r^i}\right|_0
\in T_0 \mathbb{R}^n.
\end{align*}
Let $\tilde{c} : (-\varepsilon, \varepsilon) \to \phi(U)$ be given by
$$
  \tilde{c}(t)
= (a^1 t, \dots, a^n t).
$$
We can check that $\tilde{c}(0) = 0$ and
$\tilde{c}^\prime(0) = \phi_{\ast, p}(X_p)$.
Then we let
$c : (-\varepsilon, \varepsilon) \to U \subset M$ be given by
$c = \phi^{-1} \circ \tilde{c}$ so that
$$
  c(0)
= \phi^{-1}(\tilde{c}(0))
= \phi^{-1}(0)
= p
$$
and
\begin{align*}
   c^\prime(0)
&= c_{\ast,0}
     \left(
       \left.\frac{\dif}{\dif t}\right|_0
     \right)
 = (\phi^{-1} \circ \tilde{c})_{\ast, 0}
     \left(
       \left.\frac{\dif}{\dif t}\right|_0
     \right) \\
&= \phi^{-1}_{\ast, 0}
     \left(
       tilde{c}_{\ast, 0}
       \left(
         \left.\frac{\dif}{\dif t}\right|_0
       \right)
     \right)
 = \phi^{-1}_{\ast, 0}(\tilde{c}^\prime(0)) \\
&= \phi^{-1}_{\ast, 0}(\phi_{\ast,p}(X_p))
 = X_p.
\end{align*}
\end{proof}

Note that we could have instead started by defining tangent vectors at
a point $p$ to be equivalence classes of germs at $p$. The definition
using derivations is nice because it immediately provides the vector
space structure to the tangent space. Curves are however nicer to work
with. We usually evaluate the differential using the following rule.

\begin{prop}
Let $F: N \to M$ be a smooth map between manifolds and
$p \in N$, $X_p \in T_p N$. If $c(t)$ is a curve in $N$ with
$c(0) = p$ and $c^\prime(0) = X_p$, then
$$
  F_{\ast, p}(X_p)
= (F \circ c)^\prime(0) \in T_{F(p)} M.
$$
\end{prop}
\begin{proof}
\begin{align*}
   F_{\ast, p}(X_p)
&= F_{\ast, p}(c^\prime(0))
 = F_{\ast, p}
     \left(
       c_{\ast,0}
       \left(
         \left.\frac{\dif}{\dif t}\right|_0
       \right)
     \right) \\
&= (F \circ c)_{\ast, 0}
     \left(
       \left.\frac{\dif}{\dif t}\right|_0
     \right) \\
&= (F \circ c)^\prime(0).
\end{align*}
\end{proof}

\begin{remark}
This proposition is particularly useful for working with maps between
Euclidean spaces and/or submanifolds of Euclidean spaces.
\end{remark}

\begin{xmpl}
$\mathrm{GL}_n(\mathbb{R})$ is the complement of the set of $n \times
n$ matrices with determinant 0, and so this sits inside $\mathbb{R}^{n
  \times n}$ as an open subset. If $g \in \mathrm{GL}_n(\mathbb{R})$,
then $T_g \mathrm{GL}_n(\mathbb{R}) = T_g \mathbb{R}^{n \times n}$.
Writing tangent vectors in terms of a basis for the tangent space gives
$$
\sum
  \beta^i_j \frac{\partial}{\partial a^1_j}
$$
and so we have a map
$T_g \mathbb{R}^{n \times n} \to \mathbb{R}^{n \times n}$ given by
$$
\sum
  \beta^i_j \frac{\partial}{\partial a^1_j}
\mapsto
  \beta^i_j.
$$
We have a left-multiplication map
$$
L_g: \mathrm{GL}_n(\mathbb{R}) \to \mathrm{GL}_n(\mathbb{R})
$$
given by $A \mapsto gA$. We will evaluate
$(L_g)_{\ast, \id} : T_I \mathrm{GL}_n(\mathbb{R}) \to T_g \mathrm{GL}_n(\mathbb{R})$.
Let $X \in T_I \mathrm{GL}_n(\mathbb{R}) = \mathbb{R}_{n \times
  n}$. We want a curve
$c: (-\varepsilon, \varepsilon) \to \mathrm{GL}_n(\mathbb{R})$ with
$c(0) = \id$ and $c^\prime(0) = X$, so let $c(t) = I \to t X$. For
small $t$ this is invertible since the determinant is continuous. Then
\begin{align*}
   (L_g)_{\ast, \id}(X)
&= (L_g \circ c)^\prime(0)
 = (gI + tgX)^\prime(0)
 = gX,
\end{align*}
so the derivative of left-multiplication is again left-multiplication.
\end{xmpl}

\section{Submanifolds}
Level sets of smooth functions $F: N^n \to M^m$, with $n \geq m$, are
generically smooth submanifolds of $N^n$ of dimension $n - m$. Here a
level set is a subset of the form $F^{-1}(\{c\})$.

\subsection{Embedded (regular) submanifolds}

\begin{defn}[Embedded submanifold]
A subset $S$ of a $C^\infty$ $n$-manifold $N^n$ is a
\emph{$k$-dimensional embedded} or \emph{regular submanifold} if
locally, $S$ is the common zero set of $n - k$ coordinate functions in
some chart of $N^n$. That is,
$\forall p \in S$, $\exists (U, \phi) = (U, x^1, \dots, x^n)$ a chart
about $p$ and $n - k$ distinct indices $i_1, \dots, i_{n-k} \in \{ 1,
\dots, n \}$ such that
$$
U \cap S = \{ u \in U \vert x^{i_1}(u) = \cdots = x^{i_{n-k}}(u) = 0 \}.
$$
\end{defn}

\begin{defn}[Codimension]
If $S$ is a $k$-dimensional submanifold of an $n$-manifold, then $n -
k$ is the \emph{codimension} of $S$.
\end{defn}

\begin{remark}
  \begin{enumerate}
    \item{
      Without loss of generality,
      $$
      U \cap S = \{ u \in U \vert x^{k+1}(u) = \cdots = x^n(u) = 0 \}
      $$
      because permutations of the indices
      $\sigma: \mathbb{R}^n \to \mathbb{R}^n$ given by
      $\sigma(r^1, \dots, r^n) = (r^{\sigma(1)}, \dots,
      r^{\sigma(n)})$ is a diffeomorphism.
    }
    \item{
      The chart $(U, \phi)$ in the definition is said to be
      \emph{adapted to $S$}.
    }
    \item{
      A prototypical embedded submanifold is given by
      $$
      R^k \simeq S = \{ (r^1, \dots, r^k, 0, \dots, 0) \}
      $$
      in $\mathbb{R}^n$. Indeed any embedded submanifold looks locally
      like $R^k$ embedded in $R^n$ in this way.
    }
    \item{
      These objects are called submanifolds because they possess a
      submanifold structure.
      Under the subspace topology $S$ inherits the Hausdorff and
      second-countable properties.
      Let $\mathcal{U} = \{ (U, \phi) \}$ be the collection of charts
      adapted to $S$ that cover $S$. Let $i: \mathbb{R}^k \to
      \mathbb{R}^n$ be the inclusion map
      $(r^1, \dots, r^k) \mapsto (r^1, \dots, r^k, 0, \dots 0)$, and
      $\pi: \mathbb{R}^n \to \mathbb{R}^k$ be the projection map
      $(r^1, \dots, r^n) mapsto (r^1, \dots, r^k)$. Given
      $(U, \phi) \in \mathcal{U}$, define
      $\phi_S : U \cap S \to \mathbb{R}^k$ by $\phi_S = \pi \circ
      \phi$. We claim that
      $\{ (U \cap S, \pi \circ \phi) \vert (U, \phi) \in \mathcal{U}
      \}$ forms a smooth atlas. The inverses
      $\phi_S^{-1} = \phi^{-1} \circ i$ give a homeomorphism, so we
      have a topological manifold. Next we wish to check smooth
      compatibility. Let $(U \cap S, \phi_S)$ and $(V \cap S, \psi_S)$
      be charts in our proposed atlas. Then
      $$
      \psi_S \circ \phi_S^{-1} = \pi \circ \psi \circ \phi^{-1} \circ i
      $$
      is a composition of smooth maps and is hence smooth.
    }
  \end{enumerate}
\end{remark}

\begin{xmpl}
An example of a space which is not a submanifold is the ``topologist's
sine curve'', which is connected but not path-connected. As a set,
$$
S = \{ (0, y) \vert -1 < y < 1 \}
\cup \{ (x, \sin\left(\frac{1}{x}\right)) \vert 0 < x < 1 \}.
$$
Because the curve accumulates on itself nontrivially at the origin, no
neighborhood of the origin looks like a copy of $\mathbb{R}^1$ in $\mathbb{R}^2$.
\end{xmpl}

\begin{defn}[Regular points, critical points]
Let $F: N^n \to M^m$ be a smooth map. We say that
$p \in N$ is \emph{regular} if
$F_{\ast, p}: T_pN^n \to T_{F(p)}M^m$ is surjective, which requires
that $n \geq m$ by the rank-nullity theorem. A point that is not
regular is called \emph{critical}. We define the \emph{critical set}
to be the set $\mathrm{Crit}(F)$ of all critical points of $F$. Points
that lie in the image of the critical set $F(\mathrm{Crit}(F))$ are
called \emph{critical values}, and points in
$M - F(\mathrm{Crit}(F))$ are called \emph{regular values}.
A point $c \in M$ is a regular value if either
\begin{enumerate}
  \item{$F^{-1}(c) = \varnothing$.}
  \item{
    If $F^{-1}(c)$ is nonempty, then each $p \in F^{-1}(c)$ is a
    regular point in $N$.
  }
\end{enumerate}
\end{defn}

\begin{theorem}[Sard's theorem]
If $F: N^n \to M^m$ is a smooth map, then almost every $m \in M$ is a
regular value, i.e. $F(\mathrm{Crit}(F))$ has Lebesgue measure zero in
$M$ -- i.e. in every chart it has Lebesgue measure zero.
In particular, if we take any open set in $M^m$, it will contain
regular values.
\end{theorem}

\begin{theorem}[Regular value theorem]
Let $F: N^n \to M^m$ be $C^\infty$, and $c \in M^m$ be a regular
value. Then if $S = F^{-1}(c)$ is nonempty, $S$ is an embedded
submanifold of $N^n$ of dimension $k = n - m$, i.e. of codimension $m$.
\end{theorem}

\begin{proof}
Let $S = F^{-1}(c) \neq \varnothing$, and $c$ be a regular value. We
need to exhibit an adapted chart for each $p \in S$. Let $(V, \psi)$
be a chart in $M^m$ centered at $c$ with coordinates $y^1, \dots,
y^m$, taking $V$ to $\mathbb{R}^m$ with local coordinates $r^1, \dots,
r^m$. Let $p \in F^{-1}(c)$. Since $F$ is continuous, $F^{-1}(V)$ is
open, so for any $p \in F^{-1}(c)$ we have a neighborhood $U$ centered
at $p$, and a coordinate function $\phi$ on $U$ since $S$ is an open
set belonging to the manifold $N^n$.
\begin{align*}
   U \cap S
&= \{ u \in U \vert F(u) = c \} \\
&= \{ u \in U \vert \psi \circ F(u) = 0 \} \\
&= \{ u \in U \vert r^i \circ \psi \circ F(u) = 0 \} \\
&= \{ u \in U \vert y^i \circ F(u) = 0 \} \\
&= \{ u \in U \vert F^i(u) = 0 \},
\end{align*}
so that $U \cap S$ is ``cut out'' by the $m$ component functions of
$F$.

We conclude by altering the chart $(U, \phi)$ to obtain a chart
$(\bar{U}, \Phi)$ for which $F^i$ are coordinate functions of $\Phi$.
In coordinates,
$$
  F_{\ast, p}
= \mathrm{Jac}(F)|_p
= \left[\frac{\partial F^i}{\partial x^j}(p)\right]
$$
is surjective by assumption. Therefore the Jacobian has an $m \times
m$ minor with nonzero determinant. Without loss of generality let this
be the minor consisting of the first $m$ columns. Define
$\Phi: U \to \mathbb{R}^n$ by
$\Phi(u) = (F^1(u), \dots, F^m(u), x^{m+1}, \dots, x^n)$. Then
$\mathrm{Jac}(\Phi)|_p$
can be decomposed as a block-diagonal matrix with the full-rank
$m \times m$ block in the top left and $I_{(n - m) \times (n - m)}$ in
the bottom right, which has nonzero determinant.
\end{proof}

\subsection{Lie groups}
\begin{xmpl}
  \begin{enumerate}
    \item{
      Spheres have a smooth manifold structure. Let
      $F: \mathbb{R}^{n+1} \to \mathbb{R}$ be the map
      $$
        F(x^1, \dots, x^{n+1})
      = \sum_{i=1}^n (x^i)^2 \geq 0.
      $$
      Since this is a map between Euclidean spaces, the differential
      of $F$ is exactly its Jacobian
      \begin{align*}
         \mathrm{Jac}(F)
      &= \left[\begin{array}{c c c c}
           \frac{\partial F}{\partial x^1} &
           \frac{\partial F}{\partial x^2} &
           \dots &
           \frac{\partial F}{\partial x^3}
         \end{array}\right] \\
      &= \left[\begin{array}{c c c c}
           2x^1 &
           2x^2 &
           \dots &
           2x^n
         \end{array}\right]
      \end{align*}
      so this linear transformation fails to be surjective only at
      $\mathbf{0}$. Hence, every $r^2 > 0$ is a regular value, and so
      for each $r^2$ we have a submanifold $S^n_{r^2} = F^{-1}(r^2)$.
    }
    \item{
      The \emph{special linear group} $\mathrm{SL_n(\mathbb{R})}$ is
      the group of $n \times n$ matrices with determinant 1. We have a
      map
      $\det: \mathrm{GL}_n(\mathbb{R}) \to \mathbb{R} \setminus \{ 0
      \}$, and $\mathrm{GL}_n(\mathbb{R}) = \det^{-1}(1)$.

      Let $A = [a^i_j] \in \mathrm{GL}_n(\mathbb{R})$. Again this is a
      map between Euclidean spaces, and
      \begin{align*}
         \mathrm{Jac}(\det)
      &= \left[ \frac{\partial \det}{\partial a^i_j} \right].
      \end{align*}
      Using cofactor expansion in the $i$th row and writing
      $C^i_j$ for the $i,j$th minor of $A$,
      \begin{align*}
         \det(a^i_j)
      &= (-1)^{i+1} a_1^i \det(C_1^i)
       + \cdots
       + (-1)^{i+n} a_n^i \det(C_n^i),
      \end{align*}
      and $a_1^i, \dots, a_n^i$ do not appear in any of the minors
      $C^i_1, \dots, C^i_n$. Therefore the Jacobian fails to surject
      if and only if
      $
        \frac{\partial \det}{\partial a^i_j}
      = (-1)^{i+j} \det(C_j^i)
      $
      which is true if and only if $\det C_j^i = 0$, which is true if
      and only if $\det A = 0$, a contradiction since
      $A \in \mathrm{GL}_n(\mathbb{R})$. Therefore
      $\mathrm{SL}_n(\mathbb{R})$ is a $C^\infty$ $(n^2 -
      1)$-manifold. Since it has a group structure as well, this is a
      Lie group. (In general closed subsets of Lie groups that are
      submanifolds are Lie groups).
    }
    \item{
      Consider the extra-special linear group
      $\mathrm{SL}_2(\mathbb{R})$. This is of dimension $3 = 2^2 - 1$
      and has three one-dimensional subgroups:
      \begin{align*}
         K
      &= \left\{
           \left.
             \left(
               \begin{array}{r r}
                 \cos \theta & -\sin \theta \\
                 \sin \theta &  \cos \theta
               \end{array}
             \right)
           \right\vert
           \theta \in \mathbb{R}
         \right\}, \\
         N
      &= \left\{
           \left.
             \left(
               \begin{array}{r r}
                 1 & s \\
                 0 & 1
               \end{array}
             \right)
           \right\vert
           s \in \mathbb{R}
         \right\} \\
      &\simeq \mathbb{R}, \\
         A
      &= \left\{
           \left.
             \left(
               \begin{array}{r r}
                 e^t & 0 \\
                 0   & e^{-t}
               \end{array}
             \right)
           \right\vert
           t \in \mathbb{R}
         \right\} \\
      &\simeq \mathbb{R}.
      \end{align*}
      However $K \simeq S^1$, since there is an isomorphism from
      $\mathbb{R} / 2 \pi \mathbb{Z}$.

      There is a theorem of Iwasawa that
      $K \times N \times A \to \mathrm{SL}_2(\mathbb{R})$ given by
      $(k, n, a) \mapsto k n a$ is a diffeomorphism (not a
      group homomorphism) so that
      $\mathrm{SL}_2(\mathbb{R})$ is diffeomorphic to the handlebody
      $S^1 \times \mathbb{R}^2$.

      We can also see that the tangent space at the identity
      $T_I \mathrm{SL}_2(\mathbb{R})$ has basis
      \begin{align*}
        \left.\frac{\partial}{\partial \theta}\right|_{\theta = 0}
      &= \left[
           \begin{array}{r r}
             0 & -1 \\
             1 &  0
           \end{array}
         \right], \\
         \left.\frac{\partial}{\partial s}\right|_{s = 0}
      &= \left[
           \begin{array}{r r}
             0 &  1 \\
             0 &  0
           \end{array}
         \right], \\
      \left.\frac{\partial}{\partial t}\right|_{t = 0}
      &= \left[
           \begin{array}{r r}
             1 &  0 \\
             0 & -1
           \end{array}
         \right],
      \end{align*}
      all of which have trace zero. We can show that zero is a regular
      value of the trace on the set of $2 \times 2$ matrices, and
      indeed $T_I \mathrm{SL}_2(\mathbb{R})$ is the submanifold of all
      trace 0 matrices.
    }
    \item{
      The Euclidean/orthogonal group $\mathcal{O}(n)$ is
      \begin{align*}
         \mathcal{O}(n)
      &= \{ A \in M^{n \times n} \vert \| A v \| = \| v \| \forall v
         \} \\
      &= \{ A \in \mathrm{GL}_n(\mathbb{R}) \vert
            \| A v \| = \| v \|
         \} \\
      &=  \{ A \in \mathrm{GL}_n(\mathbb{R}) \vert
            \langle A v, A w \rangle = \langle v, w \rangle
         \} \\
      &= \{ A \in \mathrm{GL}_n(\mathbb{R}) \vert
            v^T A^T A w = v^T w
         \} \\
      &= \{ A \in \mathrm{GL}_n(\mathbb{R}) \vert
            A^T A = \mathrm{Id}
         \} \\
      &= \{ A \in \mathrm{GL}_n(\mathbb{R}) \vert
            \text{$A$'s columns form a basis}
         \} \\
      &= \{ A \in \mathrm{GL}_n(\mathbb{R}) \vert
            \text{$A$'s rows form a basis}
         \} \\
      \end{align*}
      where $\|v\| = \langle v, v \rangle^{\frac{1}{2}}$ and
      $\langle v, w \rangle = v^T w$. This is a subgroup (since
      $(AB)^T(AB) = B^T A^T A B = B^T B = I$) of the
      isometry group of Euclidean space.
      The full isometry group of $\mathbb{R}^n$ is
      $\mathcal{O}(n) \ltimes \mathbb{R}^n$, which includes
      translations by an arbitrary vector.

      The map $f: \mathrm{GL}_n(\mathbb{R}) \to
      \mathrm{GL}_n(\mathbb{R})$ given by $f(A) = A^T A$ has values in
      the symmetric matrices $\mathrm{Sym}_n$. The dimension of
      $\mathrm{Sym}_n$ is $\sum_{i=1}^n i = \frac{n(n+1)}{2}$, and
      $T_B \mathrm{Sym}_n = \mathrm{Sym}_n$ for any $B \in
      \mathrm{Sym}_n$.

      We wish to know when the pushforward
      $$
        f_{\ast, A} :
            T_A \mathrm{GL}_n(\mathbb{R})
        \to T_{f(A)} \mathrm{Sym}_n(\mathbb{R})
      $$
      is surjective. (This is really a map from
      $\mathrm{Mat}_n(\mathbb{R}) \to \mathrm{Sym}_n(\mathbb{R})$.
      Let $X \in T_A(\mathrm{GL}_n(\mathbb{R})$. Let
      $c(t) = A + tX$, so that $c(0) = A$ and $c^\prime(0) = X$. Then
      $f_{\ast, A}(X) = (f \circ c)^\prime(0)$ and
      \begin{align*}
         (f \circ c)(t)
      &= c(t)^T c(t) \\
      &= (A + tX)^T(A + tX) \\
      &= A^TA + t(A^T X + X^T A) + t^2 X^T X,
      \end{align*}
      so $f_{\ast, A}(X) = A^T X + X^T A$. To check surjectivity, we
      wish to exhibit a matrix $X \in \mathrm{Mat}_n$ such that
      $A^T X + X^T A = Y$ for any $Y$. Let
      $X = \frac{1}{2} (A^T)^{-1} Y$ so that
      \begin{align*}
         A^T X + X^T A
      &= \frac{1}{2}
           A^T (A^T)^{-1} Y
       + \frac{1}{2}
           Y^T((A^T)^{-1})^T \\
      &= \frac{1}{2} Y + \frac{1}{2} Y^T \\
      &= Y.
      \end{align*}
      Therefore $f_{ast, A}$ is surjective, so
      $\mathcal{O}(n) = f^{-1}(I)$ is a $C^\infty$ manifold of
      dimension
      $$
        n^2 - \frac{(n+1)n}{2}
      = \frac{n^2 - n}{2}
      = \frac{n(n-1)}{2}.
      $$

      The orthogonal group is not connected -- it has two connected
      components. To see this consider that
      $\det: \mathcal{O}(n) \to \{ \pm 1 \}$,
      since
      $$
        \det(A A^T)
      = \det (A) \det (A^T)
      = (\det A)^2
      = \det(I) = 1,
      $$
      and note that this is a surjective group homomorphism.
      The special orthogonal group is given by
      $$
        \mathrm{SO}(n)
      = \ker(\det)
      = \{A \vert A^TA = I, \det A = 1 \}.
      $$
      Both
      $\mathrm{SO}(n)$ and $\det^{-1}(-1)$ (not a group) are clopen,
      so these are two connected components.

      Furthermore $T_I \mathcal{O}(n) = T_I \mathrm{SO}(n)$. Let
      $X \in T_I \mathcal{O}(n)$ and define a curve by
      $c(t) = I + tX \subset \mathrm{SO}(n)$, which has $c(0) = I$ and
      $c^\prime(0) = X$. Then
      $$
      c(t)^T c(t) = I
      $$
      and differentiating this equation gives
      $$
        c^\prime(0)^T
        c(0)
      + c(0)^T c^\prime(0)
      = 0
      $$
      i.e. $X^TI + I^T X = X^T + X = 0$, so $X^T = -X$. Therefore
      $T_I\mathrm{SO}(n)$ is the space of skew-symmetric $n \times n$
      real matrices, which also has dimension
      $\frac{n(n-1)}{2}$.
    }
  \end{enumerate}
\end{xmpl}

In a Lie group, the tangent space at the identity element is called
the \emph{Lie algebra}.

\begin{tabular}{c | c}
  $n$ &
  $\dim \mathcal{O}(n) = \dim \mathrm{SO}(n) = \frac{n(n-1)}{2}$ \\
  \hline \\
  1 & 0 ($\mathcal{O}(n) \simeq \mathbb{Z}_2 = \langle x \mapsto -x \rangle$) \\
  2 & 1 ($\mathrm{SO}(2)$ is $2 \times 2$ rotation matrices) \\
  3 & 3 ($\mathrm{SO}(3) \simeq \mathbb{R}P^3$)
\end{tabular}
Also observe that $\mathcal{O}(n)$ and $\mathrm{SO}(n)$ are compact,
since they are closed (as preimages $f^{-1}(I)$ of functions into
$\mathbb{R}^{n^2}$) and bounded: a matrix in
$\mathcal{O}(n)$ has columns that form an orthonormal basis of
$\mathbb{R}^n$. In particular, the columns each have Euclidean norm 1.
Therefore each entry $a^i_j$ satisfies $-1 \leq a^i_j \leq 1$.

Note that since columns of $\mathrm{SO}(3)$ form a basis,
$\mathrm{SO}(3)$ acts transitively on $S^2$, so
$S^2 = \mathrm{SO}(3) / \mathrm{SO}(2)$, and more generally
$S^n = \mathrm{SO}(n+1) / \mathrm{SO}(n)$.

Now we show a sketch of how $\mathrm{SO}(3) \simeq \mathbb{R}P^3$.
Take the $n$-disk
$\mathbb{D}^n = \{ x \in \mathbb{R}^n \vert \| x \| \leq 1 \}$ and its
boundary $\partial D^n = S^{n-1}$. To exhibit a homeomorphism (in fact
a diffeomorphism exists) we therefore wish to identify
$\mathrm{SO}(3)$ with points in $\mathbb{D}^3$ with opposite boundary
points identified.

Let $A \in \mathrm{SO}(3)$. Then 1 is an eigenvalue of $A$, since
$p(\lambda) = \det(A - \lambda I)$ is a cubic polynomial, and
therefore has at least one real root by the intermediate value
theorem, since the leading term looks like $cx^3$ and so some zero of
$p$ is real. If this root is one we are done. If not, it is -1, so
$A$ flips a line $L$ and leaves fixed as a set the perpendicular plane
$L^\perp$. Therefore the restriction $A\restrict_{L^\perp}$ is a
linear isometry with eigenvalue -1. Geometrically this map reflects
$\L^\perp$ through $L$, and so the vector at their intersection is
unchanged, so this has an eigenvalue 1.

Let $L$ be a line fixed by $A$. Then $A\restrict_{L^\perp}$ acts as a
rotation because $\det(A\restrict_{L^\perp}) = 1$. Use this rotation
to put an arrow on $L$ satisfying the right hand rule, and let $v$ be
a unit vector in that direction. This is only well-defined provided
the angle of rotation is not an integer multiple of $\pi$. We have
then defined a map $\mathrm{SO}(3) \to \mathbb{D}^3(pi) / \sim$ given
by
$A \mapsto \theta v$.

\section{Rank of a map}
Let $F: N^n \to M^m$ be a smooth map and $p \in N^n$.
\begin{defn}[Rank]
The \emph{rank of $F$ at $p$} is the rank of its differential
$F_{\ast,p} : T_pN \to T_{f(p)}M$.
\end{defn}

The maximal allowable rank of an
$m \times n$ matrix is $\min\{m, n\}$.
\begin{itemize}
  \item[($n = m$)]{
    If $F$ has full rank at $p$, then $F_{\ast,p}$ is invertible, so
    $F$ is a local diffeomorphism at $p$.
  }
  \item[($n < m$)]{
    If $F$ has maximal rank $n$ at $p$, then $F_{\ast,p}$ is injective. We
    say that $F$ is an \emph{immersion} at $p$. The prototype of this
    case is the inclusion map $\mathbb{R}^n \hookrightarrow \mathbb{R}^{n+k}$.
  }
  \item[($n > m$)]{
    If $F$ has maximal rank $m$ at $p$, then $F_{\ast, p}$ is
    surjective. We say that $F$ is a \emph{submersion} at $p$. The
    prototype of this case is the projection map
    $\mathbb{R}^n \to \mathbb{R}^{n-k}$.
  }
\end{itemize}

We will show that when $F$ is an immersion/submersion that local
coordinates can be identified such that locally $F$ looks like these
prototype functions.

\begin{remark}
If $F_{\ast, p}$ has maximal rank, then $F_{\ast, u}$ does as well for
all $u$ sufficiently close to $p$, i.e. having maximal rank is an open
condition. Choose coordinates $(U, x^1, \dots, x^n)$ at $p$
and $(V, y^1, \dots, y^m)$ at $f(p)$. Let
$F^i = y^i \circ F : U \to \mathbb{R}$ be component functions. Then
$F_{\ast, p} = \left[\frac{\partial F^i}{\partial x^j}\right]$ has
rank $k$. Then there is some $k \times k$ minor with nonzero
determinant. Any minor of larger size has zero determinant,
 and the determinant is a continuous function.
\end{remark}

We wish to give normal forms for maps with constant rank (potentially
not maximal) on an open neighborhood.

\begin{theorem}
Let $O \subset \mathbb{R}^n$ be an open set and
$F: O \to \mathbb{R}^m$ be smooth.
If $F \in C^\infty$ and has constant rank $k$ on a neighborhood of
some point $p \in O$, then $F$ has a \emph{normal form} in a
neighborhood of $p$, i.e. there exist neighborhoods
$\tilde{U}$ of $p$ and $\tilde{V}$ of $F(p)$ and
$C^\infty$ maps
$\tilde{\phi}: \tilde{U} \to \tilde{\phi}(\tilde{U}) \subset \mathbb{R}^n$
that carries $p \mapsto 0$ and
$\tilde{\psi} : \tilde{V} \to \tilde{\psi}(\tilde{V})$ that carries
$F(p) \mapsto 0$ such that
$$
  (\tilde{\psi} \circ F \circ \tilde{\phi}^{-1})
    (r^1, \dots, r^n)
= (r^1, \dots, r^k, 0, \dots, 0).
$$
\end{theorem}

\begin{remark}
Tu's appendix has a proof of the special case $n = m = 2$ and $k = 1$,
which models more or less the proof of the preimage theorem. Tu also
explains that this is equivalent to the inverse function theorem.
\end{remark}

Assuming this statement for Euclidean spaces, we deduce its extension
to manifolds as follows.

\begin{theorem}[Constant Rank Theorem]
Let $F: N^n \to M^m$ be a smooth map with constant rank in some
neighborhood of a point $p$. Then there exist coordinate charts
$(U, \phi)$ and $(V, \psi)$ centered at $p$, $F(p)$ such that
$\psi F \phi^{-1}: \phi(U) \to \psi(V)$ is given by
$$
  \psi F \phi^{-1}
    (r^1, \dots, r^n)
= (r^1, \dots, r^k, 0, \dots 0).
$$
\end{theorem}

\begin{proof}
Choose coordinates $(U_1, \phi_1)$ and $(V_1, \psi_1)$ centered at
$p$, $F(p)$ such that $F$ has constant rank $k$ on $U_1$. Then
$\psi_1 F \phi_1^{-1}$ is a map between Euclidean spaces with a rank
that coincides with the rank of $F$ (by the chain rule). Therefore
from the theorem above we have charts on $\mathbb{R}^n$ and
$\mathbb{R}^m$ given by
$(\tilde{U}, \tilde{\phi})$ and $(\tilde{V}, \tilde{\psi})$ such that
$$
  \tilde{\psi_1} \psi_1 F \circ \phi_1^{-1} \tilde{\phi_1}^{-1}
    (r^1, \dots, r^n)
=   (r^1, \dots, r^k, 0, \dots 0)
$$
so we let $\phi = \tilde{\phi_1} \circ \phi_1$ and
$\psi = \psi_1 \circ \tilde{\psi_1}$.
\end{proof}

\begin{corol}
If $F: N^n \to M^m$ is an immersion at $p$ then there exist
coordinates centered at $p$, $F(p)$ such that $F$ locally has the form
of an inclusion $(r^I) \mapsto (r^I, 0, \dots, 0)$. If $F$ is a
submersion at $p$ then there are charts that make it locally look like
a projection.
\end{corol}

\begin{defn}
We say a map $F$ is a submersion (immersion) if it is at all points $p
\in N^n$.
\end{defn}

\begin{corol}
If $F$ is a submersion, then it is an open map.
\end{corol}

\begin{proof}
It is sufficient to show that a linear projection $\pi$ is an open map.
\end{proof}

\begin{theorem}[Regular value theorem for maps of constant rank]
Let $F: N^n \to M^m$ and $c \in M^m$ with $F^{-1}(c) \neq
\varnothing$. If $F$ has constant rank $k$ on a neighborhood of
$F^{-1}(c)$, then the level set $S = F^{-1}(c)$ is an embedded
(regular) smooth submanifold of $N^n$ of codimension $k$.
\end{theorem}

\begin{proof}
Let $p \in F^{-1}(c)$. We wish to show that $p$ has an adapted
coordinate chart $U$, i.e. one for which $S \cap U$ is defined by the
common vanishing of $k$ coordinate functions. Let $(U, \phi)$ and
$(V, \psi)$ be centered charts at $p$, $F(p)$ as in the proof of the
constant rank theorem, so that
$\psi F \phi^{-1} = (r^1, \dots, r^k, 0, \dots 0)$. Then
\begin{align*}
   U \cap S
&= \{ u \in U ~\vert~ F(u) = c \} \\
&= \{ u \in U ~\vert~ \psi F(u) = 0 \} \\
&= \{ u \in U ~\vert~ r^i \psi F(u) = 0 \forall i = 1, \dots, m \} \\
&= \{ u \in U ~\vert~ \phi(u) \text{ has its first $k$ coordinates 0}
   \} \\
&= \{ u \in U ~\vert~ (r^i \circ \phi)(u), i=1, \dots, k \} \\
&= \{ u \in U ~\vert~ x^i(u) = 0, i=1, \dots, k \}.
\end{align*}
\end{proof}

We are now interested in cases when the image of a smooth map $F: N
\to M$ is or is not an embedded submanifold of $M$.

\begin{enumerate}
  \item{
    If $F$ is surjective, its image is an embedded submanifold.
  }
  \item{
    If $N$ is compact, $M$ is connected, and $F$ is a submersion then
    $F$ is surjective.
  }
  \item{
    For example, $F: \mathbb{R} \mapsto \mathbb{R}^2$ given by
    $f(t) = (t^2, t^3)$ (the \emph{cuspoidal cubic}, so called since
    $y^2 = x^3$) is \emph{not} an embedded submanifold. We see that
    $$
      F_{ast, t}
        \left(
          \frac{\dif}{\dif t}
        \right)
    = \left[
        \begin{array}{c}
          2t \\ 3t^2
        \end{array}
      \right]
    $$
    which is injective except when $t = 0$, i.e. $F$ is an immersion
    except at $t = 0$.
  }
  \item{
    A non-injective immersion is given by
    $F(t) = (t^2 - 1, t^3 - t)$, which has differential
    $$
      F_{ast, t}
        \left(
          \frac{\dif}{\dif t}
        \right)
    = \left[
        \begin{array}{c}
          2t \\ 3t^2 - 1
        \end{array}
      \right]
    $$
    which is never zero. The locus of points
    $x = t^2 - 1$ is given by
    $$
      y^2
    = (t^3 - t)^2
    = (tx)^2
    = x^3 + x^2.
    $$
    This is called the \emph{nodal cubic}, and is an immersion but not
    an embedding because the origin has no neighborhood that looks
    like a copy of $\mathbb{R}$ in $\mathbb{R}^2$.
  }
  \item{
    An injective immersion whose image is not an embedded submanifold
    is the map $F: (-\infty, 1) \to \mathbb{R}^2$ given by
    $$
      F(t)
    = \left(
        \frac{t^2 - 1}{t^2 + 1},
        \frac{t(t^2 - 1)}{t^2 + 1}
      \right)
    $$
  }
\end{enumerate}
