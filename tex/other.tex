\begin{theorem}[Brouwer's fixed point theorem]
Any continuous map $F: \bar{\mathbb{D}}^n \to \bar{\mathbb{D}}^n$ has at least one fixed point.
\end{theorem}

\begin{theorem}
A continuous map $F: \mathbb{R}^n \to \mathbb{R}^n$ is continuous and
$\exists D > 0$ such that $\|F(x) - x\| \leq D$ (``bounded
distortion'')
for all $x$, then $F$ is onto.
\end{theorem}

\begin{proof}
Let $y \in \mathbb{R}^n$ and $G: \mathbb{D}(0, D + \| y \|) \to \mathbb{R}^n$
be given by $G(x) = x - F(x) + y$. Then the image of $G$ lies in
$\mathbb{D}(0, D + \| y \|)$ since
$$
     \| G(x) \|
=    \| x - F(x) + y \|
\leq \| x - F(x) \| + \| y \|
\leq D + \| y \|.
$$
Let $x_0 \in \mathbb{D}(0, D + \| y \|)$ be a fixed point. Then
$x_0 = G(x_0) = x_0 - F(x_0) + y$ so $F(x_0) = y$.
\end{proof}

\begin{theorem}[Hairy ball theorem]
The sphere $S^n$ admits a nowhere-vanishing continuous vector field if
and only if $n$ is odd.
\end{theorem}

\subsection{Orientation on products}
Let $(M, [\omega_M]), (N, [\omega_N])$ be oriented manifolds. Then we
have projections
$\pi_M : M \times N \to M$ and $\pi_N : M \times N \to N$. Then
we have a form of top degree given by
$$
\omega_{M \times N} = \pi_M^\ast \omega_M \wedge \pi_N^\ast \omega_N.
$$
We claim that this form is nowhere zero. Let $(m, n) \in M \times N$.
Then $\phi : T_{(m, n)} M \times N \to T_m M \oplus T_n N$ given by
$\phi(v) = ((\pi_M)_\ast(v), (\pi_N)_\ast(v))$ is an isomorphism.

Let $v_1, \dots, v_{\dim M}$ given by $v_i = (x_i, 0)$, where $\{x_i\}$ is a
basis for $T_p M$, and similarly $w_i = (0, y_i)$
$i = 1, \dots, \dim N$, where $\{y_i\}$ is a
basis for $T_n N$. Also set $V_i = \phi_{i}^{-1} v_i$,
$W_i = \phi_i^{-1} w_i$. Then
$$
  \omega_{M \times N}(V_1, \dots, V_{\dim M}, W_1, \dots, W_{\dim N})
= \omega_M((\pi_M)_\ast V_1, \dots, (\pi_M)_\ast V_{\dim M}) \cdot
  \omega_N((\pi_N)_\ast W_1, \dots, (\pi_N)_\ast W_{\dim N})
\neq 0.
$$

\subsection{Cylinder over $M$}
Consider the cylinder $M \times I$, where $I = [0, 1]$, and
$M$ (\emph{without} boundary) is equipped with an orientation form
$\omega_M$. Let
$M_t = \pi_I^{-1}(t)$. We have
$\partial(M \times I) = M_0 \coprod M_1$. Denote the inclusion map for
each $M_t$ by $i_t: M_t \to M$, and the slice map $T_t : M \to M_t$
given by $m \mapsto (m, t)$. Note that
$$
  \pi_M \circ i_1 \circ T_1
= \mathrm{id}_M
= \pi_M \circ i_0 \circ T_0.
$$

The orientation on $M \times I$ is given by the form
$$
  \omega_{M \times I}
= \pi_M^\ast \omega_M \wedge \pi_I^\ast \dif t
$$
which induces an orientation on the boundary.
Consider the isomorphism
$\phi: T_{(m, s)} (M \times I) \to T_m M \oplus T_s I$ and note that
$\frac{\partial}{\partial t} \triangleq \phi^{-1}(0, \frac{\dif}{\dif
  t})$.
On $M_1$ this is an outward-pointing vector field, so we define
$$
  X_p
= \left\{
    \begin{array}{c c}
      \frac{\partial}{\partial t}, & \quad p \in M_1 \\
     -\frac{\partial}{\partial t}, & \quad p \in M_0 \\
    \end{array}
  \right.,
$$
which gives an outward-pointing vector field on $\partial(M \times
I)$.
Then we have an orientation form
\begin{align*}
   \omega_1
&= i_{\frac{\partial}{\partial t}} (i_1)^\ast \omega_{M \times I} \\
&= i_{\frac{\partial}{\partial t}} (i_1)^\ast(\pi_M^\ast \omega_M
\wedge \pi_I^\ast \dif t) \\
&= (-1)^{\dim M} i_1^\ast \pi_M^\ast \omega_M
\end{align*}
on $M_1$ and
\begin{align*}
   \omega_0
&= (-1)^{\dim M + 1} i_0^\ast \pi_M^\ast \omega_M.
\end{align*}

\begin{lemma}
$T_1$ is orientation preserving if and only if $T_0 : M \to M_0$ is
orientation reversing.
\end{lemma}
\begin{proof}
We want to compare
$$
  \int_M T_1^\ast \omega_1
= \pm \int_{M_1} \omega_1.
$$
But
$$
  \int_{M_0} T_1^\ast \omega_1
= (-1)^{\dim M}\int_{M_0} T_1^\ast i_1^\ast \pi_M^\ast \omega_M
= (-1)^{\dim M}\int_{M_0} \omega_M
$$
and
$$
  \int_{M_0} T_0^\ast \omega_0
= (-1)^{\dim M} \int_{M_0} \omega_M
= -\int_{M_1} T_1^\ast \omega_1.
$$
\end{proof}

\subsection{Homotopy}

\begin{defn}[Homotopy]
Let $X, Y$ be spaces and $f, g: X \to Y$ be continuous.
A \emph{homotopy} from $f$ to $g$ is a continuous map
$H: X \times I \to Y$ with $H(x, 0) = f(x)$ and
$H(x, 1) = g(x)$. We can think of $H$ as a path in the space of maps
$X \to Y$, where $H_t : X \to Y$ given by $H_t(x) = H(x, t)$ gives a
one-parameter family of maps.
\end{defn}

\begin{prop}
Let $M$, $N$ be compact oriented $n$-manifolds without boundary, and
$f, g : M^n \to N^n$ be diffeomorphisms. If there exists a $C^\infty$
homotopy $H : M \times I \to N$ of $f$ to $g$, then $f$ and $g$ both
preserve or both reverse orientation.
\end{prop}
\begin{proof}
Let $\omega_M$, $\omega_N$ be the orientation forms on $M$ and $N$ and
$H: M \times I \to N$ be a smooth homotopy. Let
$\alpha = H^\ast \omega_N \in \Omega^n(M^n \times I)$. Let
$i = i_0 \coprod i_1 : \partial (M \times I) = M_0 \coprod M_1  \to M
\times I$. By Stokes' theorem,
$$
  \int_{M \times I} \dif \alpha
= \int_{\partial(M \times I)} i^\ast \alpha.
$$
But
$$
  \int_{M \times I}
    \dif \alpha
= \int_{M \times I}
    \dif (H^\ast \omega_N)
= \int_{M \times I}
    H^\ast \dif \omega_N
= \int_{M \times I}
    H^\ast 0
= 0
$$
and
\begin{align*}
  \int_{\partial(M \times I)}
    i^\ast \alpha
&= \int_{M_1}
     i_1^\ast \alpha
 + \int_{M_0}
     i_0^\ast \alpha \\
&= \int_{M_1}
      i_1^\ast H^\ast \omega_N \\
 + \int_{M_0}
     i_0^\ast H^\ast \omega_N \\
&= \pm \int_M
     T_1^\ast i_1^\ast H^\ast \omega_N \\
&= \mp \int_M
     T_0^\ast i_0^\ast H^\ast \omega_N \\
&= \pm
     \left[
       \int_M
         T_1^\ast i_1^\ast H^\ast \omega_N
     - \int_M
         T_0^\ast i_0^\ast H^\ast \omega_N
     \right].
\end{align*}
Now $T_1^\ast i_1^\ast H^\ast = (H \circ i_1 \circ T_1)^\ast$
and $\forall m \in M$,
$$
  H(i_1(T_1(m)))
= H(i, (m, 1))
= H(m, 1)
= g(m)
$$
and
$$
  H(i_0(T_0(m))) = f(m)
$$
so
$$
  \int_M g^\ast \omega_N
= \pm \int_N \omega_N
= \int_M f^\ast \omega_N
$$
since $\int_{\partial(M \times I)} i^\ast \alpha = 0$. Therefore $g$
and $f$ must have the same action on orientation.
\end{proof}

\begin{proof}[Hairy ball theorem]
\begin{itemize}
  \item[($\impliedby$)]{
    $$
            S^{2n + 1}
    \subset \mathbb{R}^{2n + 2}
    = \mathbb{R}^2 \times \cdots \times \mathbb{R}^2
    $$
    with $n + 1$ copies. Let
    $X: S^n \to \mathbb{R}^{2n + 2}$
    be given by
    $$
      X(\langle x_1, y_1, \dots, x_{n+1}, y_{n+1}\rangle)
    = \langle -y_1, x_1, \dots, -y_{n+1}, x_{n+1} \rangle.
    $$
  }
\end{itemize}
\end{proof}
