\section{Smooth Maps}

\begin{defn}[Smooth function]
A function $f: M \to \mathbb{R}$ is said to be
\emph{smooth at a point $p \in M$} if there exists a chart
$(U, \phi)$ about $p$ for which
$f \circ \phi^{-1} : \phi(U) \to \mathbb{R}$ is smooth at
$\phi(p)$. $f$ is smooth ($C^\infty$) if it is smooth at all
$p \in M$. $C^\infty(M)$ denotes the $\mathbb{R}$-algebra of smooth
functions on $M$.
\end{defn}

\begin{remark}
  \begin{enumerate}
    \item{
      Smoothness at $p$ implies continuity at $p$ since $\phi$ is a
      homeomorphism.
    }
    \item{
      We can replace ``there exists a chart'' with ``for all charts''
      containing $p$.

      Let $(V, \psi)$ be another chart about $p$. Then we have
      $$
        f \circ \psi^{-1}
      = (f \circ \phi^{-1}) \circ (\phi \circ \psi^{-1})
      $$
      is smooth.

      To check if $f$ is smooth on $M$, it suffices to check
      smoothness with respect to charts in any $C^\infty$-atlas.
    }
  \end{enumerate}
\end{remark}

\begin{defn}[Smooth map]
A continuous map $F: N^n \to M^m$ is
\emph{smooth at a point $p \in N$} if there exist charts
$(U, \phi)$ about $p$ and $(V, \psi)$ about $F(p)$ such that
$$
\psi \circ F \circ \phi^{-1} : \phi(U \cap F^{-1}(V)) \to \mathbb{R}^m
$$
is smooth at $\phi(p)$. $F$ is smooth if it is smooth at all $p \in
N$. We write $C^\infty(N, M)$ for the set of all $C^\infty$ maps from
$N$ to $M$.
\end{defn}

\begin{defn}[Diffeomorphism]
A smooth map $F : N \to M$ is a \emph{diffeomorphism} if $F$ is
bijective and its inverse is smooth. $F$ is a
\emph{local diffeomorphism at $p \in N$} if there exists a
neighborhood $U$ of $p$ such that
$F\restrict_U : U \to F(U)$ is a diffeomorphism.
\end{defn}
Diffeomorphism is an equivalence relation on smooth manifolds.

\begin{xmpl}
$F: \mathbb{R} \to S^1$ given by
$F(t) = (\cos 2 \pi t, \sin 2 \pi t)$ is a local diffeomorphism but
not a global diffeomorphism since it is not bijective.
\end{xmpl}

\begin{remark}
  \begin{enumerate}
    \item{
      Similarly, ``there exists a chart'' in our definition of smooth
      map can be replaced by ``for all charts''. Let
      $(\bar{U}, \bar{\phi})$ be a chart about $p$ and
      $(\bar{V}, \bar{\psi})$ be a chart about $F(p)$. Then
      $$
        \bar{\psi} \circ F \circ \bar{\phi}^{-1}
      = (\bar{\psi} \circ \psi^{-1})
        \circ
        (\psi \circ F \circ \phi^{-1})
        \circ
        (\phi \circ \bar{\phi}^{-1})
      $$
      is $C^\infty$ since the outer terms are smooth transitions.
    }
    \item{
      Composites of $C^\infty$ maps are $C^\infty$. Let
      $F: N \to M$ and $G: M \to P$, with
      $p$ in a chart $(U, \phi)$, $F(p)$ in a chart $(V, \psi)$,
      $G(F(p))$ in a chart $(W, \sigma)$. Then
      $$
        \sigma \circ (G \circ F) \circ \phi^{-1}
      = (\sigma \circ G \circ \psi^{-1})
        \circ
        (\psi \circ F \circ \phi^{-1})
      $$
      and each of these terms is smooth by assumption.
    }
  \end{enumerate}
\end{remark}

\begin{xmpl}
Let $M$ be a smooth manifold and $(U, \phi)$ a smooth chart. Then
$\phi: U \to \phi(U)$ is a diffeomorphism. Here $U$ has a smooth
structure inherited from $M$ as an open set, and $\phi(U)$ has a
smooth structure inherited from $\mathbb{R}^m$.

Certainly $\phi$ is bijective, and indeed gives a homeomorphism from
$U$ to a subset $\phi(U)$ of Euclidean space.
%             \phi
%       U     ->   \phi(U)
% \phi  |             | Id
%       v     Id      v
%     \phi(U) ->   \phi(U)
\end{xmpl}

\begin{defn}[Lie group]
A \emph{Lie group} is a $C^\infty$ manifold that is a group such that
the map
$m: G \times G \to G$ given by $m(g_1, g_2) = g_1 g_2$ is $C^\infty$
and $i: G \to G$ given by $i(g) = g^{-1}$ is $C^\infty$.
\end{defn}

\begin{xmpl}
  \begin{enumerate}
    \item{
      $(\mathbb{R}^n, +)$ is a smooth manifold.
    }
    \item{
      $(\mathbb{C}^\times, \cdot)$ is a smooth manifold, which has as
      a subgroup $(S^1, \cdot)$.
    }
    \item{
      $G_1 \times G_2$ is a Lie group.
    }
    \item{
      The tori $\prod S^1$ are Lie groups.
      The torus $\mathbb{R}^2 / \mathbb{Z}^2$. If you take a line
      through the origin with irrational slope, then quotienting by
      this line gives $\mathbb{T}^2 / \pi(L)$ which is a Cantor set
      (i.e. not closed).
    }
    \item{
      $$
        \mathrm{GL}_n(\mathbb{R})
      = \Set{ A \in \mathbb{R}^{n \times n} | \det A \neq 0 }
      = \mathbb{R}^{n \times n} - \{ \det A = 0 \}
      $$
      is an open subset of $\mathbb{R}^{n \times n}$ since
      $\ker(\det)$ is closed because $\det$ is continuous (it is a
      polynomial). Furthermore the product of elements in
      $\mathrm{GL}_n(\mathbb{R})$ is a continuous map (since it is a
      polynomial map). Inversion is a continuous map since
      $$
        (A^{-1})^i_j
      = \frac{1}{\det A} (-1)^{i + j} \det(C^i_j)
      $$
      is a rational function (Cramer's theorem).
    }
  \end{enumerate}
\end{xmpl}

\section{Inverse function theorem}
We have a criterion for a smooth map to be a local diffeomorphism.
Let $F : \mathbb{R}^n \to \mathbb{R}^m$ be smooth. Let $p \in
\mathbb{R}^n$. Then we can define the Jacobian matrix
$$
[J(F)|_p]^i_j = \left[\frac{\partial F^i}{\partial r^j}(p)\right]
$$
where $r^i$ are local coordinates in $\mathbb{R}^n$ and we
write $F^i = r^i \circ F : \mathbb{R}^n \to \mathbb{R}$.

\begin{theorem}[Inverse function theorem (Euclidean maps)]
$F$ is a local diffeomorphism at $p \in \mathbb{R}^n$ if and only if
$J(F)|_p$ is invertible, i.e. $n = m$ and $\det(J(F)|_p) \neq 0$.
\end{theorem}

\subsection{Inverse function theorem for manifolds}
Let $f : M^m \to \mathbb{R}$ and $(U, \varphi)$ be a chart about
$p \in M$. We have standard coordinates $r^i$ on $\mathbb{R}^m$
and so coordinates $x^i = r^i \circ \phi$ on $U$. Define
\begin{align*}
   \frac{\partial f}{\partial x^i}(p)
&= \left.\frac{\partial}{\partial x^i}\right|_p f
 = \frac{\partial (f \circ \phi^{-1})}{\partial r^i}(\phi(p))
 = \left.\frac{\dif}{\dif t}\right|_{t = 0} (f \circ \phi^{-1})(c(t))
\end{align*}
where $c(t)$ is a line in $\phi(U)$. We can pull this back to $U$ via the
coordinate functions and define the derivative in terms of its
derivatives in Euclidean space.

\begin{defn}[Jacobian]
The Jacobian $J(F)|_p$ is the matrix
$\left[\frac{\partial F^i}{\partial x^j}(p)\right]$.
\end{defn}

\begin{prop}
$J(F)|_p = J(\psi F \phi^{-1})|_{\phi(p)}$.
\end{prop}

\begin{proof}
  \begin{align*}
     \frac{\partial F^i}{\partial x^j}(p)
  &= \frac{\partial (F^i \circ \phi^{-1})}{\partial r^j}(\phi(p))
   = \frac{\partial (y^i \circ F \circ \phi^{-1})}
          {\partial r^j}(\phi(p)) \\
  &= \frac{\partial (r^i \circ (\psi \circ F \circ \phi^{-1}))}
          {\partial r^j}(\phi(p)) \\
  &= \frac{\partial (\psi F \phi^{-1})^i}
          {\partial r^j}(\phi(p)).
  \end{align*}
\end{proof}

\begin{corol}[Inverse function theorem]
A map $F: N \to M$ is a local diffeomorphism at $p \in N$ if and only
if $J(F)$ is invertible.
\end{corol}

\begin{proof}
  We saw last time that the coordinate functions $\phi : U \to \mathbb{R}^n$ and
  $\psi : V \to \mathbb{R}^m$ are diffeomorphisms onto their images.
\end{proof}
